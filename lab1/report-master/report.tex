\include{settings}

\begin{document}	% начало документа

% Титульная страница
\begin{titlepage}	% начало титульной страницы
	
	\begin{center}		% выравнивание по центру
		
		\large Санкт-Петербургский политехнический университет Петра Великого\\
		\large Институт компьютерных наук и технологий \\
		\large Высшая школа интеллектуальных систем и суперкомпьютерных технологий\\[6cm]
		% название института, затем отступ 6см
		
		\huge Разработка графических приложений\\[0.5cm] % название работы, затем отступ 0,5см
		\large Отчет по лабораторной работе №1\\[0.1cm]
		\large Операции над гистограммами\\[5cm]
		
	\end{center}
	
	
	\begin{flushright} % выравнивание по правому краю
		\begin{minipage}{0.25\textwidth} % врезка в половину ширины текста
			\begin{flushleft} % выровнять её содержимое по левому краю
				
				\large\textbf{Работу выполнила:}\\
				\large Ольшанская В.Р.\\
				\large {Группа:} 3540901/01502\\
				
				\large \textbf{Преподаватель:}\\
				\large Абрамов Н. А.
				
			\end{flushleft}
		\end{minipage}
	\end{flushright}
	
	\vfill % заполнить всё доступное ниже пространство
	
	\begin{center}
		\large Санкт-Петербург\\
		\large \the\year % вывести дату
	\end{center} % закончить выравнивание по центру
	
\end{titlepage} % конец титульной страницы

\vfill % заполнить всё доступное ниже пространство


% Содержание
\include{ToC}


\section{Цель работы}
Первичное ознакомление с библиотекой opencv, получение навыков построения гистограмм и изучение операций над ними. Улучшения контраста изображения при помощи операций над гистограммами.


\section{Программа работы}
1. Построить гистограмму черно-белого изображения.

2. Сделать линейное растяжение гистограммы и построить соответствующее изображение. 

3. Сделать эквализацию гистограммы и построить соответствующее изображение.


\section{Ход выполнения работы}

\subsection{Построение гистограммы изображения}

\begin{figure}[H]
	\begin{center}
		\includegraphics[scale=0.7]{Lenna.png}
		\caption{Оригинальное изображение (в цвете)} 
		\label{pic:hist_orig} % название для ссылок внутри кода
	\end{center}
\end{figure}

\begin{figure}[H]
	\begin{center}
		\includegraphics[scale=0.8]{hist_org.JPG}
		\caption{Сравнение гистограмм} 
		\label{pic:hist_linear} % название для ссылок внутри кода
	\end{center}
\end{figure}
\begin{figure}[H]
	\begin{center}
		\includegraphics[scale=0.3]{rez.JPG}
		\caption{Сравнение изображений} 
		\label{pic:hist_linear} % название для ссылок внутри кода
	\end{center}
\end{figure}

\subsection{Сравнение способов улучшения на искусственно созданном изображении}

\begin{figure}[H]
	\begin{center}
		\includegraphics[scale=0.3]{rez2.jpg}
		\caption{Сравнение оригинального и улучшенных изображений} 
		\label{pic:hist_eq} % название для ссылок внутри кода
	\end{center}
\end{figure}
\begin{figure}[H]
	\begin{center}
		\includegraphics[scale=0.7]{hist_last.JPG}
		\caption{Сравнение гистограмм оригинального и улучшенных изображений} 
		\label{pic:hist_eq} % название для ссылок внутри кода
	\end{center}
\end{figure}

\subsection{Сравнение способов улучшения}
\begin{figure}[H]
	\begin{center}
		\includegraphics[scale=0.3]{rez3.jpg}
		\caption{Сравнение оригинального и улучшенных изображений} 
		\label{pic:hist_eq} % название для ссылок внутри кода
	\end{center}
\end{figure}
\begin{figure}[H]
	\begin{center}
		\includegraphics[scale=0.7]{hisl_last_art.JPG}
		\caption{Сравнение гистограмм оригинального и улучшенных изображений} 
		\label{pic:hist_eq} % название для ссылок внутри кода
	\end{center}
\end{figure}

\section{Выводы}
Если на гистограмме исходного изображения два локальных максимума, расположенных по краям изображения, линейное растяжение гистограммы практически не изменяет гистограмму, а при эквализации возможно появление артефактов в области интенсивности исходных локальных максимумов.
\end{document}
