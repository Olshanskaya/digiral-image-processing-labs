\include{settings}
\usepackage{amsmath}

\begin{document}	% начало документа

% Титульная страница
\begin{titlepage}	% начало титульной страницы
	
	\begin{center}		% выравнивание по центру
		
		\large Санкт-Петербургский политехнический университет Петра Великого\\
		\large Институт компьютерных наук и технологий \\
		\large Высшая школа интеллектуальных систем и суперкомпьютерных технологий\\[6cm]
		% название института, затем отступ 6см
		
		\huge Разработка графических приложений\\[0.5cm] % название работы, затем отступ 0,5см
		\large Отчет по лабораторной работе №1\\[0.1cm]
		\large Операции над гистограммами\\[5cm]
		
	\end{center}
	
	
	\begin{flushright} % выравнивание по правому краю
		\begin{minipage}{0.25\textwidth} % врезка в половину ширины текста
			\begin{flushleft} % выровнять её содержимое по левому краю
				
				\large\textbf{Работу выполнила:}\\
				\large Ольшанская В.Р.\\
				\large {Группа:} 3540901/01502\\
				
				\large \textbf{Преподаватель:}\\
				\large Абрамов Н. А.
				
			\end{flushleft}
		\end{minipage}
	\end{flushright}
	
	\vfill % заполнить всё доступное ниже пространство
	
	\begin{center}
		\large Санкт-Петербург\\
		\large \the\year % вывести дату
	\end{center} % закончить выравнивание по центру
	
\end{titlepage} % конец титульной страницы

\vfill % заполнить всё доступное ниже пространство


% Содержание
\include{ToC}


\section{Цель работы}
Ознакомиться со способами выделения контуров при помощи операторов Робертса, Превитта и Собеля.


\section{Программа работы}
1. Выделить границы тремя способами на естественном изображении с ярковыраженными границами

2. Выделить границы тремя способами на искусственном изображении с ярковыраженными границами 

3. Сравнить результаты работы разных операторов и сделать выводы.


\section{Ход выполнения работы}

Операторы Робертса 

$\begin{pmatrix}
	1 & 0\\ 
	0 & -1
\end{pmatrix}$ 
$\begin{pmatrix}
	0 & 1\\ 
	-1 & 0
\end{pmatrix}$ 


Операторы Собеля 

$\begin{pmatrix}
	-1 & -2 & -1\\ 
	0 & 0 & 0 \\
	1 & 2 & 1
\end{pmatrix}$ 
$\begin{pmatrix}
	-1 & 0 & 1\\ 
	-2 & 0 & 2 \\
	-1 & 0 & 1
\end{pmatrix}$ 


Операторы Превитта 

$\begin{pmatrix}
	-1 & -1 & -1\\ 
	0 & 0 & 0 \\
	1 & 1 & 1
\end{pmatrix}$ 
$\begin{pmatrix}
	-1 & 0 & 1\\ 
	-1 & 0 & 1 \\
	-1 & 0 & 1
\end{pmatrix}$ 

Магнитуда считается как $\sqrt{x^2 + y^2}$



\subsection{Выделить границы тремя способами на естественном изображении с ярковыраженными границами}

\begin{figure}[H]
	\begin{center}
		\includegraphics[scale=0.5]{zebra.jpg}
		\caption{Оригинальное изображение в цвете} 
		\label{pic:hist_orig} % название для ссылок внутри кода
	\end{center}
\end{figure}
\begin{figure}[H]
	\begin{center}
		\includegraphics[scale=0.2]{robrts_all_2.jpg}
		\caption{Выделение границ при помощи операторов Робертса} 
		\label{pic:hist_orig} % название для ссылок внутри кода
	\end{center}
\end{figure}
\begin{figure}[H]
	\begin{center}
		\includegraphics[scale=0.2]{Sobel_all_2.jpg}
		\caption{Выделение границ при помощи операторов Собеля} 
		\label{pic:hist_orig} % название для ссылок внутри кода
	\end{center}
\end{figure}
\begin{figure}[H]
	\begin{center}
		\includegraphics[scale=0.2]{prewit_all_2.jpg}
		\caption{Выделение границ при помощи операторов Превитта} 
		\label{pic:hist_orig} % название для ссылок внутри кода
	\end{center}
\end{figure}



\subsection{Выделить границы тремя способами на искусственном изображении с ярковыраженными границами}
\begin{figure}[H]
	\begin{center}
		\includegraphics[scale=0.5]{lines.JPG}
		\caption{Оригинальное изображение} 
		\label{pic:hist_orig} % название для ссылок внутри кода
	\end{center}
\end{figure}
\begin{figure}[H]
	\begin{center}
		\includegraphics[scale=0.25]{robrts_all_1.jpg}
		\caption{Выделение границ при помощи операторов Робертса} 
		\label{pic:hist_orig} % название для ссылок внутри кода
	\end{center}
\end{figure}
\begin{figure}[H]
	\begin{center}
		\includegraphics[scale=0.25]{Sobel_all_1.jpg}
		\caption{Выделение границ при помощи операторов Собеля} 
		\label{pic:hist_orig} % название для ссылок внутри кода
	\end{center}
\end{figure}
\begin{figure}[H]
	\begin{center}
		\includegraphics[scale=0.25]{prewit_all_1.jpg}
		\caption{Выделение границ при помощи операторов Превитта} 
		\label{pic:hist_orig} % название для ссылок внутри кода
	\end{center}
\end{figure}

\subsection{Сравнить результаты работы разных операторов и сделать выводы}

\begin{figure}[H]
	\begin{center}
		\includegraphics[scale=0.2]{all_sravnenie_2.jpg}
		\caption{Сравнение выделения границ операторами Робертса, Собеля и Превитта на естественном изображении} 
		\label{pic:hist_orig} % название для ссылок внутри кода
	\end{center}
\end{figure}
\begin{figure}[H]
	\begin{center}
		\includegraphics[scale=0.25]{all_sravnenie_1.jpg}
		\caption{Сравнение выделения границ операторами Робертса, Собеля и Превитта на искусственном изображении} 
		\label{pic:hist_orig} % название для ссылок внутри кода
	\end{center}
\end{figure}


\section{Выводы}
Оператор Робертса работает не так качественно как операторы Собеля и Превитта. Оператор Собеля довольно похож на оператор Превитта, а видоизменение заключается в использовании весового коэффициента 2 для средних элементов. Это увеличенное значение используется для уменьшения эффекта сглаживания за счет придания большего веса средним точкам. Этот эффект виден только при большом увеличении изображения
\end{document}
